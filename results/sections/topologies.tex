\documentclass[../main.tex]{subfiles}

\title{Topologies}
\author{Jack Wile}
\date{}

\begin{document}
Here we will record results pertaining to the topologies one can put on an Op$^*$-algebra. Our notions of correctability and privacy
for quantum channels of unbounded operators will be stated in terms of the seminorms generating a locally convex topology. However, it is 
not immediately clear which topology is best for the task of proving the desired results, which are most satisfying mathematically or physically,
and how each topology relates to one another.

\begin{definition}
	An O*-algebra equipped with a locally convex topology for which multiplication is seperately continuous is called
	a locally convex algebra. If the involution is also continuous, we say that it is a locally convex $\ast$-algebra.
\end{definition}


\subsection{The Uniform Topology}

Here we will outline the results of Lassner concerning their so-called uniform topology. Before we can define what that is, we require
a topology on the domain of an Op*-algebra.

Let $\ms{A}$ be an Op*-algebra. We denote by $\mc{T}_\ms{A}$ the coarsest locally convex topology on $\ms{D} = \mc{D}(\ms{A})$
for which every operator $A \in \ms{A}$ is a continuous linear mapping of $(\ms{D}, \mc{T}_\ms{A}) \to (\ms{D}, \ip{\cdot}{\cdot})$.
This topology is given by the seminorms $\norm{\psi}_A = \|A\psi\|$, where $A \in \ms{A}$ on $\ms{D}$. This is stronger than the Hilbert 
space topology since $\ms{A}$ is unital. Though we assumed the codmain to have the Hilbert space topology, every $A \in \ms{A}$ is 
continuous if it also has $\mc{T}_\ms{A}$.

\begin{theorem}

	Given an Op*-algebra $\ms{A}$ on a pre-Hilbert space $\ms{D}$ we have the following.
	
	\begin{enumerate}
	\item Every linear operator $A \in \ms{A}$ is a continuous operator of $(\ms{D}, \mc{T}_\ms{A}) \to (\ms{D}, \mc{T}_\ms{A})$.
	\item If every operator in $\ms{A}$ is bounded, then $\mc{T}_\ms{A}$ coincides with the Hilbert space topology on $\ms{D}$.
	\item Given an algebraic linear basis of $\ms{A}$, $\mc{T}_\ms{A}$ is generated by those seminorms $q_A$ with $A$ in the basis.

	\end{enumerate}

\end{theorem}

The locally convex space $(\ms{D}, \mc{T}_\ms{A})$ is not in general complete. We may denote the completion of $\ms{D}$ by
$\overline{\ms{D}}$.

\begin{lemma}
	Let $\ms{A}$ be an Op*-algebra on $\ms{D} \subseteq \ms{H}$. The injection $\ms{D} \to \ms{H}$ can be extended to a continuous
	injection of $\overline{\ms{D}} \to \ms{H}$. Moreover, $\overline{\ms{D}} = \bigcap_\ms{A} \mc{D}\left(\overline{A}\right)$. 
\end{lemma}

We will now begin our overview of locally convex topologies on an Op*-algebra $\ms{A}$.

\begin{definition}
	Let $\ms{A}$ be an Op*-algebra on $\ms{D}$, $\ms{B}$  a collection of bounded sets of a locally convex space $(\ms{D}, \mc{T})$.
	We say that that $\ms{B}$ is admissible if

	\begin{enumerate}
		\item for $S \in \ms{B}$ and $A \in \ms{A}$, $A(S) \in \ms{B}$, 
		\item $\bigcup_{S \in \ms{B}} S$ is dense in $\mc{H}(\ms{D})$,
		\item for $S_1, S_2 \in \ms{B}$, $S_1 \cup S_2 \in \ms{B}$.
	\end{enumerate}



\end{definition}

These so-called admissable systems give class of locally convex topologies through the topology $\mc{T}_\ms{A}$ on 
$\ms{D}$.


\begin{definition}

	Let $\ms{A}$ be an Op*-algebra on $\ms{D}$ and $\ms{B}$ an admissable system of bounded sets of $(\ms{D}, \mc{T}_\ms{A})$.
	We define the topology $\mc{T}^\ms{B}$ by the seminorms \[ q_{B, S}(A) = \sup_{\psi \in S} \norm{A\psi}_B, \ S \in \ms{B}, B \in \ms{A}. \] 
	Similarly, we define the topology $\mc{T}_\ms{B}$ by the seminorms \[\norm{A}_S = \sup_{\phi, \psi \in S} |\ip{\phi}{A\psi}|, \  S \in \ms{B}. \]
	If $\ms{B}$ is the system of all bounded sets on $(\ms{D}, \mc{T}_\ms{A})$, we write these topologies as $\mc{T}^\ms{D}$ and $\mc{T}_\ms{D}$, 
	respectively.

\end{definition}

Under any admissable system $\ms{B}$ with respect to an Op*-algebra $\ms{A}$ on $\ms{D}$, $(\ms{A}, \mc{T}^\ms{B})$ and $(\ms{A}, \mc{T}_\ms{B})$
are locally convex spaces. Moreover, we have the following;

\begin{theorem}

If $\ms{A}$ is an Op*-algebra, then $(\ms{A}, \mc{T}^\ms{B})$ is a locally convex algebra, and $(\ms{A}, \mc{T}_\ms{B})$ is a locally
convex $\ast$-algebra.

\end{theorem}

Lassner calls this the uniform topology, as a generalization of the operator norm in the bounded case due to the next theorem.

\begin{theorem}

	If every operator $A$ of an Op*-algebra $\ms{A}$ on $\ms{D}$ is bounded, then $\mc{T}^\ms{D} = \mc{T}_\ms{D} = \mc{T}_{\norm{\cdot}}$
	where $\mc{T}_{\norm{\cdot}}$ is the operator norm topology. $\mc{T}^\ms{D}$ is finer than $\mc{T}_\ms{D}$ and they are equivalent if and 
	only if multiplication is $\mc{T}_\ms{D}$-continous.

\end{theorem}


There is something of a converse to the first part of the previous theorem.

\begin{theorem}

If there exists a norm on an Op*-algebra $\ms{A}$ on $\ms{D}$ defining a stronger topology than $\mc{T}_\ms{D}$, then every operator $A \in \ms{A}$
is continuous.

\end{theorem}

Finally, Lassner gives a lemma characterizing the topology $\mc{T}_\ms{D}$ without reference to $\mc{T}_\ms{A}$.

\begin{lemma}

	Let $\ms{A}$ be an Op*-algebra over $\ms{D}$. The topology $\mc{T}_\ms{D}$ is defined by all seminorms 
	\[\|A\|_S = \sup_{\phi, \psi \in S}|\ip{\phi}{A\psi}|,\] where $S$ is an arbitrary subset of $\ms{D}$ such that
	the written supremum exists for any $A \in \ms{A}$.


\end{lemma}


\subsection{The $\beta$ Topology}

Here we will discuss another topology given by Lassner which emerges from a dual pairing familiar to
bounded operator algebras.

\begin{definition}
	
By $\ms{T}_1(\ms{D})$ we refer to those nuclear operator $\rho \in \ldag{\ms{D}}$ for which
\begin{enumerate}
	\item $A\rho B$ and $A \rho^\dagger B$ are nuclear for all $A, B \in \ldag{\ms{D}}$.
	\item $\ip{AB}{\rho C} = \ip{B\rho}{CA}$ for all $A, B, C \in \ldag{\ms{D}}$.
\end{enumerate}
Where $\ip{\rho}{A} = \text{Tr}(\rho A)$ and $\ip{A}{\rho} = \text{Tr}(A\rho)$.

\end{definition}

Then $\ip{\rho}{A}$ gives a dual pairing for $\pair{\ms{T}_1(\ms{D})}{\ldag{\ms{D}}}$. We denote by $\beta$ the topology 
induced by this dual pairing; the locally convex topology given by the seminorms \[\beta_S(A) = \sup_{\rho \in S} |\ip{\rho}{A}|.\]
Where $S$ is a weakly bounded subset of $\ms{T}_1(\ms{D})$. This topology is "physically motivated" in the sense of the duality between
states $\rho$ and observables $A$. The $\beta$ topology also generalizes the uniform topology on $\ms{B}(\ms{H})$ with, in the bounded case, $\ms{T}_1(\ms{D})$
simply being all nuclear operators and the dual pairing being the familiar one. 

\subsection{The $\rho$ topology}

Let $\ms{A}$ be an Op*-algebra over $\ms{D}$ and $\{A_n\}_n \subseteq \ms{A}$
a positive, increasing sequence. Suppose that $\ms{A} = \bigcup_n \ms{A}_n$ where \[\ms{A}_n \defeq \{A \in \ms{A} ; \ \text{for any } \psi \in \ms{D} , \
|\ip{A\psi}{\psi}| \le \lambda \cdot |\ip{A_n\psi}{\psi}| \text{ for some } \lambda > 0\}.\] Then we have the norms $\norm{\cdot}_{n}: \ms{A}_n \to \mathbb{R}^+$ defined by
\[\norm{A}_n = \inf\{\lambda > 0 \ ; \ |\ip{A\psi}{\psi}| \le \lambda\cdot|\ip{A_n\psi}{\psi}| \text{ for all }\psi \in \ms{D}\}.\] Suppose further that $\ms{A}$ is the inductive
limit of the locally convex spaces $\{\pair{\ms{A}_n}{\norm{\cdot}_n}\}_n$. Since each $\ms{A}_n$ is a normed space, a neighbourhood basis of $0$ is given by unions of open balls with respect to each norm. Moreover, this topology is generated
by the following seminorms indexed by $\alpha \in (\mathbb{R}^+)^{\mathbb{N}}$ \[\rho_\alpha(A) \defeq  \inf\left\{\sum^N_{k = 1}\alpha(k)\norm{A_k}_k \ ; \ A = \sum^N_{k = 1}A_k, \ 
A_k \in \ms{A}_k\right\}.\] 

Certain examples of Op*-algebras satsify the conditions necessary to define this topology, and in many cases this topology coincides with the $\beta$ topology, such as the vital
position/momentum algebra on Schwarz space. It is unkown whether this this true for all Op*-algebras. It is easy to see that once again this topology generalizes the operator norm
in the bounded setting.


\end{document}
